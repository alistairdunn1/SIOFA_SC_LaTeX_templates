% Write the report body here
\section{Summary of recommendations and decisions from the Scientific Committee and Meeting of parties for 2013-2015}

Early MoP reports (2013-2015) did not specifically discuss reference points or harvest strategies when considering stock assessments. The only relevant comment found was in the MoP1 report \citep{MoP1} that recorded an intervention by SIODFA, which noted that the objectives of SIODFA included an appropriate harvesting regime for targeted species.

\section{Summary of recommendations and decisions from the Scientific Committee and Meeting of parties for 2016}

\subsection{The \cite{SC1} report noted the following}

Para 101: The Scientific Committee noted there is a requirement to follow the principles of the precautionary approach, whereby the absence of adequate scientific information shall not be used as a reason for postponing or failing to take conservation and management measures (Article 4(c)). Some Members noted that the Scientific Committee could recommend a prohibition on deepwater gillnets that would not necessarily preclude their future use, but that if deepwater gillnet fishing occurred it would be on the basis of having a robust ecological risk assessment undertaken, an agreed harvest strategy with clear harvest control rules.

Para 115: In discussing the management of bottom fishing in the SIOFA area (SC-01-07 (01), SC-01-07 (02), SC-01-INFO 26, SC-01-27) the Scientific Committee advises the MoP that there are several options for limiting fishing effort. Adopting effort control in SIOFA was considered prudent given the absence of quantitative assessments on the status of stocks in relation to biological reference points and an agreed harvest policy.

\begin{enumerate}
	\item limiting fishing activity in bottom and mid-water fishing in any one year to their maximum effort in any one of the reference years (which would need to be defined). Limits could be defined as total days at sea in the Agreement Area and/or vessel numbers. The Scientific Committee did not have a substantive discussion on the most appropriate effort measure.
	\item prohibiting vessels from undertaken bottom fishing in the Area outside their historical bottom fishing footprint. The term ‘bottom fishing footprint’ means a map of the spatial extent and distribution of historical bottom fishing in the Area of all vessels flagged to a particular Contracting Party, CNCP or PFE over expressed as grid blocks of 20 minute resolution over a reference period (which would need to be defined).
\end{enumerate}

Para 116: The Scientific Committee advised that Option 1 would not necessarily constrain the spatial distribution of effort. Option 2 would not constrain total effort but would constrain the spatial distribution of effort which may assist the MoP with ensuring that impacts on VMEs is minimised by preventing fishing activities from expanding into new areas. The MoP may wish to consider both options if it chooses to manage effort in terms of total effort and its spatial distribution. The MoP is advised that Scientific Committee did not discuss the implications of effort creep due to increases in fishing power of vessels on these options. The Scientific Committee did not discuss the definition of reference periods for limiting effort, suggesting this be investigated intersessionally and advice provided in future if required.

\section{Summary of recommendations and decisions from the Scientific Committee and Meeting of parties for 2017}

No discussions of harvest strategies or reference points were recorded. However, the SC2 report \citep{SC2} noted (Annex M, the SIOFA Scientific Committee Operational Work Plan 2016-2019) that the determination of biological reference points and associated development of harvest strategies was a priority. This work was included in the workplan of the SAWG, which was tasked with assisting with review of methods and outputs used for stock assessments and provide advice to the Scientific Committee on a harvest strategy and fisheries reference points for SIOFA fisheries.

\section{Summary of recommendations and decisions from the Scientific Committee and Meeting of parties for 2020}

\subsection{The \cite{SC5} report noted the following}

Para 171: No papers were provided for this agenda item [Agenda item 7.9 Harvest strategies]. The SC agreed to progress this work, in line with the agreed work plan (SC4 Report, Annex X) and reflected in the SC Operational work plan, noting the MoP6 had approved funding for this work in 2020 (MoP6 Report, Annex Q, EUR 15,000 in 2020, of a requested EUR 30,000 across two years).
No relevant comments on reference points or harvest strategies were found in the \cite{MoP7} report.

\section{Summary of recommendations and decisions from the Scientific Committee and Meeting of parties for 2021}

Paper SC-06-24 (Butterworth et al. 2021) was a SIOFA consultancy that reported on the development of harvest strategies for key target species in the SIOFA area. This paper provided the following summary:
\begin{quote}
	The Terms of Reference for this contract ask for evaluations of use of harvest strategies, and target and limit reference points, by other fishery organisations, and then for recommendations for adoption of similar approaches by SIOFA. Those practices in a number of such organisations are summarised, as are the assessments available for the three major species under harvest in the SIOFA area: alfonsino, orange roughy and Patagonian toothfish. However, for the other main species of commercial interest in this area, because only limited information is currently available, assessments (and hence reference points, and harvest strategies based on those) are not yet possible; hence, a process to move towards developing and then improving these assessments needs to be agreed. This process must include further data collection in particular, especially of catch and effort information.
	For alfonsino, orange roughy and Patagonian toothfish, the alternative merits of three different approaches need to be considered:
	\begin{enumerate}
		\item Maintaining catches at present levels (unless there is evidence of a marked downward trend in the resource) until sufficient further data become available for meaningful improvements to the existing assessments.
		\item Implementing an Fstatus-quo harvesting strategy, which varies catches up or down in proportion to the results from continued collection of some measure or index of abundance.
		\item Implementing a harvest strategy based primarily on some multiple of a proxy value of FMSY, where this in turn is based on a proxy value for a BMSY reference point whose value is informed by the most recent assessment of the resource.
	\end{enumerate}
	The choice amongst these for each of the three species separately will come down primarily to the trade-off between likely greater stability of catch limits over time under the first approach, against possibly larger catches in the short term at least under the second and third.
	For the other main, but data-poor, species in the SIOFA area, only the first approach is viable at this time, but needs to be augmented by one or more precautionary provisions. For example, the SAFE methodology might be applied to obtain some indication of whether the current catch is leading to an appreciable reduction in abundance – if so, necessitating a reduction in the present catch.
\end{quote}

\subsection{The \cite{SC6} report noted the following}

Para 117: The report on the development of harvest strategies for key target species in the SIOFA Area (\cite{butterworth_report_2021} ; also presented at SERAWG3 as SERAWG-03-10) was taken as read. The report included a summary of the use of harvest strategies, and target and limit reference points used by other fishery organisations, a summary of the assessments available for the three major species under harvest in the SIOFA Area (alfonsino, orange roughy and Patagonian toothfish), possible harvest strategy approaches for the aforementioned three major species and the pros and cons of each, and possible ways to move towards developing assessments for the other major species and consequently reference points and harvest strategies based on those assessments.

Para 122: The SC NOTED that for most other SIOFA species that are data limited, assessments and consequently reference points and harvest strategies are not yet possible to develop.

Para 123: For these SIOFA species, the SC NOTED that approach i. could be the most viable at this time, but that this would need to be augmented by one or more precautionary provisions to check whether catches were sustainable and take corrective action in the event that there were persuasive indications to the contrary. The SERAWG NOTED that this approach could be implemented, for example, by application of risk assessment across a broad suite of species using, for example, the SAFE methodology. However, unless the spatial and temporal scale of the fishery is well known, this may not be possible and other options would need to be investigated.

Para 124: The SC RECOMMENDS that the MoP note that an important associated priority is further data collection, especially more and better catch and effort information and the associated analyses of these data through space and time.

Para 125: The SC SUGGESTS that:
\begin{itemize}
	\item The utility and specifics of the three alternative approaches, as they may apply in each case, be examined before a decision on the best approach is determined.
	\item The MoP considers interim reference points for orange roughy and alfonsino as follows: Target = BMSY using a proxy of $= 0.4 B_0$, and a Limit $=0.2 B_0$ (common surrogates used in other regions). These interim reference points could be considered for SC reporting purposes and would not necessarily be appropriate for management purposes.
	\item With respect to toothfish, the MoP consider that CMM 2020/15 has an objective to “ensure collaborative and complementary arrangements are in place for D. eleginoides between SIOFA and the CCAMLR”. Accordingly, when setting reference points for toothfish, SIOFA consider the reference points adopted by CCAMLR: Target$= 0.5 B_0$, and Limit $=0.2 B_0$
	\item The MoP consider fishing fleet behaviour and fish stock structure in the development of harvest strategies for each species.
\end{itemize}

Para 126: The SC RECOMMENDS that the MoP:
\begin{itemize}
	\item Undertake analyses to determine the applicability and trade-offs between the three proposed harvest strategy approaches for each of the three species concerned, to provide an objective basis to underpin final decision making. \end{itemize}
For some approaches this will require consideration of appropriate reference points.

\subsection{The \cite{MoP8} report noted the following}

Para 130: France Territories supported the continuation of the work on harvest strategies by implementing analyses to assess the effectiveness and risks associated with the three strategies proposed in the Scientific Committee report. In view of the little knowledge on the sustainability of harvesting levels for the main species, France Territories supported the implementation of the precautionary principle when choosing the reference points. Regarding toothfish, France Territories supported the adoption of management objectives and reference points as adopted by CCAMLR.

Para 131: Australia welcomed the significant consultant report exploring the potential development of harvest strategies in SIOFA and stated that it continues to be a strong advocate of harvest strategies as a best practice in fisheries management in order to achieve SIOFA’s objectives. Australia could support the proposed interim reference points on orange roughy, alfonsino, and toothfish, but recognised that further consideration may be needed within the Scientific Committee and amongst CCPs and so Australia did not advocate for a decision on reference points at this MoP. Australia supported the recommendation on further work to examine the applicability of the three proposed harvest strategy approaches, and work to develop objectives for these fisheries.

Para 132: The Cook Islands expressed its support for the development of a harvest strategy process, noting that, while some of the issues need broader consideration, the work done so far by the Scientific Committee is a good step forward. The Cook Islands noted that, for all three stocks concerned, the scientific information available make the development of an efficient, well-balanced, and carefully thought out harvest strategy challenging, and suggested that it may be necessary to consider simpler approaches in the interim.

Para 133: The European Union welcomed the work done to progress the harvest strategy approaches and suggested that a roadmap be developed and the work be progressed further in the intersessional period before the Scientific Committee meeting to enable it to make recommendations in time for the next Meeting of the Parties.

Para 134: The European Union highlighted the need for enhanced cooperation between scientists and managers when developing harvest strategy approaches.

Para 135: The Meeting of the Parties requested the Scientific Committee to develop a roadmap for developing harvest strategies at the seventh Scientific Committee meeting and, as recommended in paragraph 126 of the SC6 report, consider analyses to determine the applicability and trade-offs between the three proposed harvest approaches for orange roughy, alfonsino, and toothfish.

\section{Summary of recommendations and decisions from the Scientific Committee and Meeting of parties for 2022}

Report SC-07-INFO-24 (\cite{butterworth_development_2022}) was a SIOFA consultancy that reported on a roadmap for the development of harvest Strategies for SIOFA. This report provided the following summary:
\begin{quote}
The Consultants past experience with conducting assessments of and/or providing management advice for SIOFA fish stocks has indicated that a key problem has been the lack of background information on the data available and how they relate to the way the fishery operates. That missing information is a key input to the assessment process, and its ability to provide reliable results. The International Whaling Commission’s “harvest strategy roadmap” is reviewed. Their first step for any stock of a “pre-assessment” process to compile the data to be used in the harvest strategy analyses and how they should be interpreted, is suggested to be an essential component of any similar SIOFA roadmap. This process should be put into practice by the appointment, for any stock for which a harvest strategy is to be developed, of a Technical Sub-Committee which would meet separately from the SIOFA Scientific Committee and report back to it. This Sub-Committee would include persons with the relevant expertise about the stock to provide this missing information and to develop ToR’s for the basis on which the harvest strategy development should proceed. Overview comments are provided about the process that would then follow. An important decision to be made is whether the harvest strategy for a specific stock is to be based on the “best assessment plus harvest control rule” approach or on Management Strategy Evaluation (MSE). A table is provided summarising the details associated with this “Technical Sub-Committee” pre-assessment component of a harvest strategy development roadmap.
\end{quote}

\begin{table}[H]
  \centering
    \caption{\label{Table1}Elements of the initial stage of a recommended harvest strategy roadmap for SIOFA, focussing on the suggested pre-assessment process (Table 1 from \cite{butterworth_development_2022}.}
    \begin{tabular}{|p{0.1\linewidth}|p{0.9\linewidth}|}
      \hline
      Step 1 & The Scientific Committee selects a stock for the potential development of a harvest strategy. Note that at any one time, probably no more than two stocks should be in process towards such development (this in the light of likely resource limitations in terms of “person-power”) \\
      \hline
      Step 2 & The Scientific Committee appoints a Technical Sub-Committee to initiate the harvest strategy development process for that stock through what is termed a “Pre-assessment”. In broad terms, the role of that Sub-Committee is to oversee the compilation of the data to be used in that process and to comment on how they are to be interpreted in developing stock assessment models and the basic hypotheses on which those models are to be based (this may extend beyond single interpretations of components of that information, and include alternatives for which sensitivities will need to be investigated). \\
      \hline
      Step 3 & The Technical Sub-Committee is to comprise of persons with the appropriate expertise to advise on the data available for the stock and how they are to be interpreted. They are to be drawn both from Scientific Committee members and from outside persons with relevant expertise. \\
      \hline
      Step 4 &  At the start of the process, the Scientific Committee should appoint likely analysts, but at that stage “preliminarily”, i.e., for participation in the activities of the Technical Sub-Committee only. \\
      \hline
      Step 5 & A primary role of the Technical Sub-Committee is to report back to the Scientific Committee when they consider that the pre-assessment process has been successfully completed to the stage that they would be prepared to recommend to the Scientific Committee that the quantitative assessment analyses by the analysts previously “provisionally” appointed can commence \\
      \hline
      Step 6 & The Technical Sub-Committee must also advise the Scientific Committee on:
      \begin{enumerate}
      \item Likely timelines for completion of the harvest strategy development.
      \item If pertinent, broad indications of likely appropriate values for target and limit reference points.
      \item ToR for the analysts who will be developing the harvest strategy.
      \item Whether to aim for a “best assessment plus harvest control rule approach” or for a full MSE harvest strategy, with the addition of further details desirably specified immediately for whichever option is preferred. 
      \end{enumerate} \\
      \hline
      Step 7 & The Scientific Committee then considers the recommendations/advice provided by the Technical Sub-Committee, and decides whether the harvest strategy development for the stock under consideration is to proceed, together with specifying the ToR for the analysts. \\
      \hline
   \end{tabular}%
\end{table}

\subsection{The \cite{SC7} report noted the following}

Para 124. The SC ENDORSED the recommendation in SC-07-INFO-12 rev 1 (Butterworth 2022):
\begin{itemize}
	\item to specify a pre-assessment process involving the appointment of a Technical Sub-Committee to oversee the collection of relevant data and to provide the interpretations of those data that are necessary before the assessment of and harvest strategy development for any stock can proceed.
	\item that subsequent harvest strategy development would be highly dependent on the reports from such Technical Sub-Committees, so it would be premature at this time to get into more details about the later stages of a harvest strategy roadmap for SIOFA.
\end{itemize}

Para 125. As the next steps, the SC RECOMMENDED:
\begin{itemize}
	\item that the Secretariat work intersessionally to prepare as much information as possible for understanding the data available on the alfonsino, orange roughy and toothfish fisheries and any potential trends in the data.
	\item that a two-day harvest strategy pre-assessment workshop be held in 2023 prior to SC8, with the participation of scientists, managers, industry representatives, and observers, to:
	\begin{enumerate}
		\item discuss the planning and implementation of the harvest strategy development roadmap.
		\item interpret the data.
		\item identify data gaps for informing a stock assessment.
		\item discuss which stocks are to be assessed.
	\end{enumerate}
	\item That the outcomes of the workshop be presented to the SC and its working groups for further discussion.
\end{itemize}

Para 126: The SC encouraged CCPs to conduct characterisations of their alfonsino, orange roughy and toothfish fisheries, and to present this information to the abovementioned workshop.

Para 170: With regard to the development of a harvest strategy roadmap, the SC RECOMMENDED the MoP:
\begin{itemize}
	\item ENDORSE the specification a pre-assessment process involving the appointment of a Technical Sub-Committee to oversee the collection of relevant data and to provide the interpretations of those data that are necessary before the assessment of and harvest strategy development for any stock can proceed.
	\item NOTE that subsequent harvest strategy development would be highly dependent on the reports from such Technical Sub-Committees so it would be premature at this time to get into more details about the later stages of a harvest strategy roadmap for SIOFA.
	\item task the Secretariat to work intersessionally to prepare as much information as possible for understanding the data available on the alfonsino, orange roughy and toothfish fisheries and any potential trends in the data.
	\item ENDORSE that a two-day harvest strategy pre-assessment workshop be held, with the participation of scientists, managers, industry representatives, and observers, to:
	\begin{enumerate}
		\item discuss the planning and implementation of the harvest strategy development roadmap.
		\item interpret the data collected intersessionally.
		\item identify data gaps for informing a stock assessment.
		\item discuss which stocks are to be assessed.
		\item develop identification guides to assist the recording of species by the vessel crew and observers.
	\end{enumerate}
	\item that the outcomes of the workshop be presented to the SC and its working groups for further discussion.
\end{itemize}

\subsection{The \cite{MoP9} report noted the following}

Para 130: The Meeting of the Parties ENDORSED the recommendations in paragraph 170 of the SC7 report regarding the development of a harvest strategy roadmap.

Para 131: The Meeting of the Parties AGREED that the holding of the harvest strategy preassessment workshop, as well as other workshops, should be done in a hybrid format to enable maximum participation, including by observers.

Para 268: The Meeting of the Parties AGREED that the joint MoP-SC workshop on harvest strategy pre-assessment will take place from 17 to 18 March 2023, the workshop on deepwater sharks in the SIOFA Area will take place from 20 to 21 March 2023, and the eighth meeting of the SC will take place from 22 to 31 March 2023, in Tenerife, Spain.
